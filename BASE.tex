\documentclass[12pt, twocolumn, a4paper]{article}
\usepackage{multicol, lipsum}
\usepackage[utf8]{inputenc}
\usepackage{cite}
\usepackage{amsmath}
\usepackage{amsfonts}
\usepackage{amssymb}
\usepackage{graphicx}
\usepackage{pslatex}
\usepackage[a4paper, left=3cm, right=2cm, top=3cm, bottom=2cm,
		headsep=1cm, footskip=2cm]{geometry}
\usepackage[brazil]{babel}

\begin{document}
	\title{SystemVerilog Vocabulary Extractor}
	\author{Filipe C. Cavalcanti\\ Leandro de S. Albuquerque\\
	Orientador: Tio Kat}
	\maketitle
	
	\section{ABSTRACT}
	
	\section{RESUMO}
	\quad Desde a criação da primeira HDL até os presentes dias, cada vez mais o desenvolvimento de sistemas digitais se assemelha e aproxima-se a codificações de programas descritos em linguagem de programação  
	
% Keywords: HDL, SystemVerilog, Software Vocabulary, Hardware Vocabulary,  
	\section{Introdução}

\quad Verilog foi uma das primeiras linguagens para descrição de hardware (HDL) a ser inventada, em meados da década de 80. O tamanho típico dos projetos era entre de 5 a 10 mil portas lógicas. 

O método de concepção dos circuitos utilizava-se de esquema gráfico, e a simulação começava a ser uma ferramenta essencial para verificação\cite{sutherland2006}. Com a evolução da tecnologia de descrição e verificação de hardware, em 2002 surge SystemVerilog. 

A partir disto, como a complexidade de sistemas digitais modernos aumentou exponencialmente, tanto que, o tamanho dos atuais projetos chega a ordem dos milhões de portas lógicas. Assim as metodologias de projetos em sistemas digitais estão evoluindo extensivamente \cite{Marc-Andre} e \cite{Hahanov2008}.

Com tal avanço, elevou-se o nível de abstração no desenvolvimento de hardware por meio de uma linguagem de descrição e verificação de hardware (HDVL), de tal forma que, o uso de ferramentas de análise de informações que antes eram somente do escopo da engenharia de software, pode ser estendido também para o desenvolvimento de sistemas digitais.

Umas das principais fontes de informações em um código fonte no âmbito da engenharia de software, é o vocabulário. Dentre suas principais utilidades listamos o seguinte:\\
\quad 1. Localização de bugs;\\
\qquad 2. Identificação de uma arquitetura;\\
\qquad 3. Métricas sobre o código fonte;\\
\qquad 4. Identificação de especialista \cite{Santos2015}.

O vocabulário de software também denominado de léxico do código em \cite{Biggers2011}, consiste no conjunto de termos repetidos ou únicos que compõem identificadores e que estão presentes no textos dos comentários\cite{Abebe2009}.

Usando os princípios da engenharia reversa como uma coleção de metodologias e técnicas capazes de realizar a extração e abstração de informações\cite{Santos2009}, propõe-se neste trabalho a extração do vocabulário pertencentes a projetos de hardware descritos em SystemVerilog, e usando a definição formal de Santos em \cite{Santos2015} sobre vocabulário de software para embasar e fundamentar o termo \textit{Hardware Vocabulary}. 

	\section{Background}
\quad Graças aos atuais projetos eletrônicos baseado em HDL, metodologias e ferramentas para simulação, síntese, verificação, modelagem física e teste pós-fabricação agora estão bem inseridos e são essenciais para designers digitais \cite{Navabi2015}. Nos últimos anos as linguagens de descrição e verificação de hardware tornaram-se tão importantes para a modelagem de sistemas digitais, quanto as linguagens de programação o são para a engenharia de software. 
	
	\subsection{Software Vocabulary}
\quad Santos em \cite{Santos2009}, define que vocabulário de código fonte compreende as cadeias de caracteres que identificam os elementos estruturais e as palavras que compõem as sentenças dos comentários de um código fonte. Dentro dos campos de estudo da engenharia de software outro termo bastante conhecido e isomorfo a \textit{Software Vocabulary} é o \textit{léxico do código} que em \cite{Biggers2011} são os elementos que nomeiam as entidades estruturais da linguagem além dos comentários escritos em linguagem natural.

 \textit{Software Vocabulary} é um multiconjunto de \textit{Strings}, i.e. uma aplicação $V:$ $\mathbb{S}\rightarrow\mathbb{N}$, que mapeia \textit{Strings} para números naturais. Elementos de um vocabulário são chamados \textit{termos}. Para qualquer termo $t$, $V(t)$ representa o número de ocorrências do termo $t$ no vocabulário $V$. Se $V(t)>0$ dizemos que $t$ é um termo do vocabulário \cite{Santos2015}.

% O lexico de um programa representa um investimento substancial para uma empresa de software, portanto, seu valor deve preservado e aumentado ao longo do tempo, para aproveitar ao máximo seus efeitos benéficos na compreensão do programa.(Antoniol)

No paradigma de programação dominante atualmente OOP (\textit{Object-Oriented Programming}), nomear os elementos estruturais da linguagem de forma concisa e representativa além de documenta-las, tem sido mais do que uma boa prática .

 No desenvolvimento de grandes sistemas o léxico ou SV (\textit{Software Vocabulary}), quando condizente ao problema, reduz o tempo de manutenção, entendimento do código e encontro de \textit{bugs}. 

O léxico de um programa representa um investimento substancial para uma empresa de software, portanto, sua importância deve preservada e elevada ao longo do tempo, para aproveitar ao máximo seus efeitos e benéficos na compreensão do programa \cite{Antoniol2007}.

	\subsection{O Hardware Como Um Software}
\quad 

	\subsection{O Que é Uma HDL?}
%\quad HDVL (Hardware Description and Verification Language), podemos abstrair como um único ambiente para design e verificação de sistemas digitais, em \cite{Flake} uma HDVL representa hardware digital em vários níveis de abstração.

\quad Uma descrição HDL(\textit{Hardware Description Language}) é uma representação precisa que pode ser usada para documentar, comunicar e simular o projeto \cite{Miller-Karlow}.

As HDLs modernas são fundamentais para o desenvolvimento de sistemas digitais, possibilitando suas descrições de forma estrutural, comportamental e nos últimos anos, seguindo conceitos básicos de orientação a objetos, fornecendo assim um mecanismo efetivo para o desenvolvimento de projetos à medida que evoluem da abstração para a realidade.

%--------------apagar essa parada!!!! DATTEBAYO
%A sintaxe pesada de uma HDL apresenta um problema no desenvolvimento de projetos, pois descreve um sistema complexo e multifacetado por meio de uma representação textual unidimencional, dificultando a abstração de \textit{\textbf{muitos}} aspetos do projeto.
	\subsection{Hardware Vocabulary} 
	\section{SystemVerilog Vocabulary Extractor}
	
\quad A fim de analisar e extrair o HV (\textit{Hardware Vocabulary}) de SystemVerlog desenvolvemos o ferramental \textit{Hardware 	 Vocabulary Tool}, e \cite{AST}
	
	
	\quad Afim de calcularmos uma porcentagem de extracão que melhor represente a eficiência do software proposto, foi elaborado um design genérico com todas as estruturas possíveis em SistemVerilog. Os resultados obtidos são apresentados na tabela abaixo:

\begin{table}[h]
\centering
\caption{Hello Word Table}
	\begin{tabular}{r|l|r}
	\hline
	posição & País & IDH\\
	\hline
	1 & Noruega        & .955 \\
	\hline
	2 & Austrália 	   & .938 \\
	\hline
	3 & EUA            &. 937 \\
	\hline
	4 & Holanda        & .921 \\
	\hline
	5 & Alemanha       & .920 \\
	\hline
	
	\end{tabular}

\end{table}

\quad Os resultados apresentados na tabela acima mostra que ...

	Foram realizados, também, outros testes com hardware \textit{opensource} obtidos em repositórios no Github. Os resultados obtidos estão expostos na tabela abaixo:
	\section{Resultados e Discussões}
	
	\begin{table}[h]
\centering
\caption{Hello Word Table}
	\begin{tabular}{r|l|r}
	\hline
	posição & País & IDH\\
	\hline
	1 & Noruega        & .955 \\
	\hline
	2 & Austrália 	   & .938 \\
	\hline
	3 & EUA            &. 937 \\
	\hline
	4 & Holanda        & .921 \\
	\hline
	5 & Alemanha       & .920 \\
	\hline
	
	\end{tabular}

\end{table}
	
  
	\bibliography{refs}
	\bibliographystyle{abbrv}
	
\end{document}